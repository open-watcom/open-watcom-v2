% !TeX root = owstl.tex

\chapter{String}

\section{Introduction}

The class template \code{std::string} provides dynamic strings of objects with
a type given by the type parameter \code{CharT}. The behavior of \code{CharT}
objects is described by a suitable traits class. By default, a specialization
of \code{std::char_traits<CharT>} is used. Specializations of
\code{std::char_traits} for both character and wide character types are part
of the library and are used without any further intervention by the
programmer.

Most of the methods in class template \code{std::string} are located in
\filepath{hdr/watcom/string.mh}. This file is also used to generate the C
library header \filepath{string.h} and the corresponding ``new style'' C++
library header \filepath{cstring}. This is accomplished by executing
\term{wsplice} over the file multiple times using different options. The
material that goes into the C++ library header \filepath{string} appears in
\filepath{string.mh} below the C library material.

The class template \code{std::char_traits} along with its specializations for
character and wide character, the definition of \code{std::string}, and
certain methods of \code{std::string} are located in
\filepath{hdr/watcom/_strdef.mh}. This file generates the header
\filepath{_strdef.h} which is not to be directly included in user programs. It
is, however, included in \filepath{string} thus completing the contents of
\filepath{string}. The reason for this separation of \filepath{string} is
because of the exception classes. The standard exception classes use strings
and yet some of the methods of string throw standard exceptions. This leads to
circular inclusions which are clearly unacceptable. To resolve this problem,
the parts of \filepath{string} that are needed by the standard exception
classes are split off into \filepath{_strdef.h}. These parts do not themselves
need the standard exceptions and so the circular reference is broken.

\section{Status}

Most of the required functionality has been implemented together with moderately
complete regression tests. There has so far been very little user feedback,
however.

The main component that is missing is the I/O support for \code{std::string}.
Implementing this component has been put on hold until the iostreams part of
the library is reworked. In the meantime, users will have to do string I/O
using C-style strings and convert them between \code{std::string}. This is a
significant issue; it is assumed that most standard programs will do I/O on
strings directly and the library doesn't currently support such programs no
matter how complete the \code{std::string} implementation itself might be.

In addition to the problem above, the template methods of \code{std::string}
have not been implemented because the compiler does not yet support template
methods sufficiently well.

\section{Design Details}

\subsection{Copy-On-Write?}

This implementation of \code{std::string} does not use a copy-on-write or a
reference counted approach. Instead, every string object maintains its own
independent representation. This was done, in large measure, to simplify the
implementation so that a reasonable \code{std::string} could be offered
quickly. However, there are a number of difficulties with making
\code{std::string} reference counted, and it is worth reviewing those issues
here.

The fundamental problem is that the \code{std::string} interface leaks
references to a string's internal representation. It could be argued that this
is a design problem with \code{std::string}. Consider the program in
Listing~\ref{lst:leakage}.

\begin{figure}[htbp]
\centering
\begin{lstlisting}[
    language=C++,
    caption={Potential Reference Leakage},
    label={lst:leakage},
    numbers=left,
    frame=single]
#include <iostream>
#include <string>

using namespace std;

int main()
{
    string s1( "Hello" ), s2;
    char &c( s1[0] );

    s2 = s1;  // s1 and s2 perhaps share representations
    c = 'x';  // Does s2 also change?
    if( s2[0] == 'x' )
       cout << "Wrong!\n";
    else
       cout << "Right!\n";
    return 0;
}
\end{lstlisting}
\end{figure}


The value semantics of \code{std::string} require that modifying one string
should not influence the value seen in another logically distinct string.
Thus, all correct implementations of \code{std::string} should produce
"Right!" for the program above.

To deal with this case properly while using reference counted strings, the
implementation must ``unshare'' the representation whenever a method is called
that leaks a reference to that representation. The method \code{operator[]} is
one example of such a method. In fact, section 21.3, paragraph 5 of the C++
standard contains explicit language regarding this issue. The standard allows
implementations to invalidate references, pointers, and iterators to the
elements of a \code{basic_string} sequence whenever, for example, the
non-const \code{operator[]} is called. However, this leads to rather
unexpected behavior in at least two respects. In particular:

\begin{itemize}
\item Accessing a string might be an O(n) operation.
\item Accessing a string might cause a \code{std::bad_alloc} exception to be
thrown.
\end{itemize}

The first issue is a concern to those doing time sensitive operations, such as
those writing embedded systems (Open Watcom's support for 16 bit 8086 targets
might be attractive to such programmers). In fact, \code{std::string} provides
a \code{reserve} method specifically to give the programmer some degree of
control over when allocations are done. Copying a string's representation
unexpectedly when a string is accessed frustrates this intention.

The second issue is a concern to those writing robust, exception safe code. To
build code that is exception safe it is important to know when exceptions
might be thrown. A savvy programmer might know that calling
\code{std::string::operator[]} might throw an exception. However, because that
is an unnatural side-effect many programmers won't be expecting it and thus
using such an implementation will be error prone. Note that on some systems,
notably Linux, the operating system will usually terminate the program when it
runs out of memory before \code{std::bad_alloc} can be thrown. However, that
is not the case on smaller, real-mode systems like DOS. Thus for Open Watcom
this issue is a concern.

In a multithreaded program reference counted strings encounter other problems.
Since Open Watcom supports a number of multithreaded targets this is also a
concern.

The C++ 1998 standard does not address the semantics of programs in the face of
multithreading. However, most programmers implicitly assume the following
behavior (described by SGI in the documentation for their STL implementation).
\todo{Should this discussion be moved to a more generic part of this document? Some
of this would be applicable to the whole OWSTL library.}

\begin{itemize}
\item Two threads can read the same object without locking. This means that if
reading an object changes its internal state, the implementation must provide
appropriate locking.
\item Two threads can operate on logically distinct objects without locking.
This means that if objects share information internally, the implementation
must provide appropriate locking.
\item If two threads operate on the same object and at least one of the
threads is modifying that object, the programmer must provide locking. This
means that the implementation does not need to protect itself from this case.
\end{itemize}


Reference counted strings must deal with both situations 1 and 2 above. This
means they must provide locks on the representations and use them when
appropriate. The problem with this is that strings are rather low level
objects and locking them is generally inappropriate. Most strings are used
entirely by one thread; locks are usually only needed on larger structures.
For example consider the following function:

\begin{lstlisting}[language=C++]
typedef std::map<std::string, std::string> string_map_t;

string_map_t global_map;

void f( )
{
    // Modify the global_map.
}
\end{lstlisting}

If more than one thread is modifying the global map it would be appropriate to
include a lock for the entire map. Locking the individual strings in the map
would most likely be too fine-grained since a single transaction might involve
updating several strings. It would be important to serialize the entire
transaction. Locking the components of the transaction separately would be
incorrect.

Yet a reference counted implementation of \code{std::string} must add locking
to the strings themselves to ensure correct behavior when apparently unrelated
strings are simultaneously modified. This would be adding a large amount of
logically unnecessary locking overhead in cases such as the one above. This
overhead can cause reference counted strings to have very poor performance
when used in a multithreaded environment. \todo{reference?} This is
particularly ironic considering that reference counting is intended to improve
string performance.

% Are there also deadlock problems in an MT program?

Concerns about the day-to-day performance of Open Watcom's non-reference
counted implementation have been partially addressed by the results of some
(minimal) benchmark tests. See \filepath{bld/bench/owstl}. These tests show
that the current performance of \code{std::string} is at least competitive
with that offered by other implementations. More complete benchmark testing is
needed to verify this result.

It is interesting to note that \term{gcc}, which at the time of this writing
(2005) uses a reference counted approach, has extraordinarily poor performance
on these benchmark tests. If this result stands up to further investigation it
would be dramatic evidence that a reference counted approach does not
automatically ensure good performance. In fact, I am led to wonder if the
\term{gcc} maintainers did any benchmark studies of their implementation or if
they just assumed that it would be fast because it is reference counted.
Either way, this highlights the importance, in my mind, of following up
performance assumptions by making real measurements on the final
implementation. One should always verify that any change designed to improve
performance actually does improve performance before committing to it.

\subsection{Design Overview}

This implementation of \code{std::string} uses a single pointer and two
counters to define the buffer space allocated for a string. One counter
measures the length of the allocated space while the other measures the number
of character units in that space that are actually used. In order to meet the
complexity requirements of the standard, \code{string} allocates more space
than it needs, increasing that amount of space by a constant multiplicative
factor whenever more is needed. This implementation uses a multiplicative
factor of two. \clarify{Note: other factors, such as 1.5, might be more
desirable; a factor of two causes somewhat inefficient memory reuse
characteristics. Reference?} The capacity of a string is always an exact power
of two. When a string is first created it is given a particular minimum size
for its capacity (currently 8) or a capacity that is the smallest power of two
larger than the new string's length, whichever is larger.

A string's capacity is never reduced in this implementation. Once a string's
capacity is increased, the memory is not reclaimed until the string is
destroyed. This was done on the assumption that if a string was once large it
will probably be large again. Not returning memory when a string's length
shrinks reduces the total number of memory allocation operations and reduces
the chances of an out of memory exception being thrown during a string
operation. However, this design choice is not particularly friendly to low
memory systems. Considering that Open Watcom targets some small machines, an
alternative memory management strategy might be worth offering as an option.
In the meantime programmers on such systems should be careful to destroy large
strings when they are no longer needed rather than, for example, just calling
\code{erase}.

\subsection{Relationship to Vector}

The \code{std::string} template is very similar in many ways to the
\code{std::vector} template. In fact, in OWSTL both implementations use a
similar representation technique and a similar memory management approach.
However, the implementation of \code{std::vector} is more complicated because
the objects in a vector need not be of a POD type (as is the case for string)
so they need to be carefully copied and initialized using appropriate methods.
In contrast, the \code{CharT} type used by \code{std::string} can be copied
and moved with low level memory copying functions (see
\code{std::char_traits}).

\section{Open Watcom Extensions}

Because of the widespread demand for case-insensitive string manipulation,
OWSTL provides a traits class that includes case-insensitive character
comparisons. An instantiation of \code{std::string}, called
\code{_watcom::istring} is provided that uses this traits class.
