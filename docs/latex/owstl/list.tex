% !TeX root = owstl.tex

\chapter{List}

\section{Introduction}

\section{Status}

Missing members:

\begin{enumerate}
    \item Err\ldots need to look through the standard.
\end{enumerate}

Completed members:

\begin{enumerate}
\item \code{explicit list( Allocator const & )}
\item \code{list( list const & )}
\item \code{~list( )}
\item \code{operator=( list const & )}
\item \code{assign( size_type, value_type const & )}
\item \code{get_allocator( ) const}
\item \code{iterator}
\item \code{const_iterator}
\item \code{reverse_iterator}
\item \code{const_reverse_iterator}
\item \code{begin( )} (+const)
\item \code{end( )} (+const)
\item \code{rbegin( )} (+const)
\item \code{rend( )} (+const)
\item \code{size( )}
\item \code{empty( )}
\item \code{front( )}
\item \code{back( )}
\item \code{push_front( value_type const & )}
\item \code{push_back( value_type const & )}
\item \code{pop_front( )}
\item \code{pop_back( )}
\item \code{insert( iterator, value_type const & )}
\item \code{erase( iterator )}
\item \code{erase( iterator, iterator )}
\item \code{swap( list& )}
\item \code{clear( )}
\item \code{remove( value_type const & )}
\item \code{splice( iterator, list & )}
\item \code{splice( iterator, list &, iterator )}
\item \code{splice( iterator list &, iterator, literator )}
\item \code{reverse( )}
\item \code{merge( list const & )}
\end{enumerate}

\section{Design Details}

\code{template <class Type, class Allocator>}\\
\code{class std::list}

\subsection{Description of a Double Linked List}

This is a data structure that is made up of nodes, where each node contains
the data, a pointer to the next node, and a pointer to the previous node. The
overall structure also knows where the first element in the list is and
usually the last. Obviously it requires two pointers for every piece of data
held in the list, but this allows movement between adjacent nodes in both
directions in constant time.

\subsection{Overview of the Class}

The class defines an internal \code{DoubleLink} structure that only holds
forward and backward pointers to itself. It then defines a \code{Node}
structure that inherits from \code{DoubleLink} and adds to it the space for
the real data (of type \code{value_type}) that is held in the list nodes. This
is done so that a special sentinel object can be created that is related to
every other node in the list, but that doesn't require space for a
\code{value_type} object. This sentinel is used by the list class to point to
the first and last elements in the list. A sentinel is useful in this case
(the alternative would just be individual first and last pointers) because it
means the insertion and deletion code does not have to check for the special
case of editing the list at the beginning and end. The sentinel is initialized
pointing to itself and is used as the reference point of the end of the list.
When an element is inserted or deleted before the end or at the beginning all
the pointer manipulation just falls out in the wash. \note{This seems to be a
good uses of sentinels.}

There are two allocators that need to be rebound for the \code{Node} and
\code{DoubleLink} types. Two allocators are needed because objects of
different types are being allocated: the node allocation allocates nodes (with
their contained \code{value_type}) while the link allocator allocates the
sentinel node of type \code{DoubleLink}.

\subsection{Inserting Nodes}

The work for the functions \code{push_front}, \code{push_back} and
\code{insert} is done by the private member \code{push}.

The member is quite simple. It allocates a \code{Node} and then tries to make
a copy of type in the memory allocated. The usual try-catch wrappings
deallocate the memory if the construction was unsuccessful. It then modifies
the pointers of the \code{Node} that was passed, the element before that node,
and the new node so that the new node is linked into place just before o. The
end of the list is signaled by the sentinel object, so if we are trying to
insert before the end o is the sentinel and everything works. If we are trying
to insert before the first node, the old node before the first is again the
sentinel, so the pointers are all valid and everything works.

\subsection{Deleting Nodes}

\subsection{Clearing All}
