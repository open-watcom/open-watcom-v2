%! TeX root = owstl.tex

\chapter{Algorithm}

\section{Introduction}

The algorithm header (truncated to \filepath{algorith} for 8.3 file name
compatibility) contains definitions of the algorithms from chapter 25 of the
C++ 1998 standard.

\section{Status}

About two thirds of the required algorithms have been implemented. For a list
of those remaining, see the Wiki.

\section{Design Details}

Most of the standard algorithms are function templates that operate on
iterators to perform some task. Each function template is quickly addressed in
the sections that follow. They are generally quite simple and looking directly
at the source may provide the best information.


A number of the algorithms come in two forms that use either \code{operator<}
or \code{operator==} (as appropriate), and in a form that uses a predicate.
The predicate form is more general. The non-predicate form can be implemented
in terms of the predicate form by using the function objects in
\code{functional}.

In theory, implementing the non-predicate forms in terms of the predicate
forms should not entail any abstraction penalty because the compiler should be
able to optimize away any overhead due to the function objects. Some
investigation was done using Open Watcom v1.5 to find out if that was true. In
fact, the compiler was able to produce essentially identical code for the
non-predicate functions that were implemented directly as it did for
non-predicate functions that were implemented in terms of the predicate
functions. However, at the call site, there was some abstraction penalty: the
compiler issued a few extra instructions to manipulate the (zero sized)
function objects.

These experiments led us to conclude that the non-predicate functions should
be implemented directly for short, simple algorithms where the extra overhead
might be an issue. For the more complex algorithms, the non-predicate forms
should be implemented in terms of the corresponding predicate forms. The extra
overhead of doing so should be insignificant in such cases and the savings in
source code (as well as the improved ease of maintenance) would make such an
approach desirable.

If the compiler's ability to optimize away the function objects improves, this
matter should be revisited.

\subsection[\texttt{*\_heap}]{\code{*_heap}}

The functions \code{push_heap}, \code{pop_heap}, \code{make_heap}, and
\code{sort_heap} support the manipulation of a heap data structure. Currently,
only versions using an explicit \code{operator<} have been implemented. The
versions taking a comparison object have yet to be created. Several
heap-related helper functions have been implemented in namespace
\code{std::_ow}. These functions are not intended for general use.

There is a compiler bug that prevents the signature of the internal
\code{heapify} function from compiling. This has been worked around by
providing the necessary type as an additional template parameter. See the
comments in \filepath{algorith.mh} for more information.

\subsection[\texttt{remove}, \texttt{remove\_if}]{\code{remove}, \code{remove_if}}

These functions ``remove'' the value that compares equal or the element at
which the predicate evaluates to true. Because iterators can't be used to
access the underlying container the element can't really be removed. These
functions instead copy elements from the right (an incremented iterator) over
the top of the element that is ``removed'' and then return an iterator
identifying the new end of the sequence. The initial implementation just
called the \code{remove_copy} and \code{remove_copy_if} functions described
below. This would perform unnecessary copies on top of the same object if
there are any values at the beginning of the container that aren't to be
removed. This could cause a performance hit if the object is large and there
are lots of objects that don't need to be removed, therefore these functions
were re-written to be independent of the \code{*_copy} versions and perform a
check for this condition.

\subsection[\texttt{remove\_copy}, \texttt{remove\_copy\_if}]{\code{remove_copy}, \code{remove_copy_if}}

This makes a copy of the elements that don't compare equal, or when the
predicate is false, starting at the location given by Output. It is a simple
while loop over the input iterator first to last, either just skipping the
element or copying it to the output.

\subsection[\texttt{unique}]{\code{unique}}

For C++ 1998 and C++ 2003 there is an open library issue regarding the
behavior of \code{unique} when non-equivalence relations are used. The
standard says that the predicate should be applied in the opposite order of
one's intuition. In particular: \code{pred(*i, *(i-1))}. This means the
predicate compares an item with its previous item.

The resolution of the open issue suggests that non-equivalence relations should
not be permitted. In any case, the standard should apply the predicate between
an item and the next item: \code{pred(*(i-1), *i)}.

The Open Watcom implementation follows the proposed resolution and thus
deliberately violates the standard. Most (all?) other implementations do the
same.

\subsection[\texttt{file\_first\_of}]{\code{find_first_of}}

There are two versions of this, one that uses \code{operator==} and one that
uses a binary predicate. There is a simple nested loop to compare each element
with each element indexed by the second iterator range.

\subsection[\texttt{find\_end}]{\code{find_end}}

There are two versions of this, one that uses \code{operator==} and one that
uses a binary predicate. The main loop executes two other loops. The first
loop finds an \code{input1} element that matches the first \code{input2}
element. When a match is found the second loop then checks to see if it is
complete match for the subsequence. If it is, the position the subsequence
started is noted and the main loop is iterated as there may be another match
later on. Note this can't search for the substring backwards as the iterators
are \code{ForwardIterator}s.

\subsection[\texttt{random\_shuffle}]{\code{random_shuffle}}

The \code{random_shuffle} template with two arguments has been implemented
using the C library function \code{rand}. However, the 1998/2003 standard is
unclear about the source of random numbers that \code{random_shuffle} should
use. There is an open library issue about this with the C++ standardization
group. See \url{http://anubis.dkuug.dk/JTC1/SC22/WG21/docs/lwg-active.html},
item \#395. The proposed resolution is to allow the implementation to use
\code{rand} without requiring it to do so (the source of random numbers is
proposed to be implementation defined).

The problem with \code{rand} in this case is that Open Watcom's implementation
of \code{rand} is limited to 16 bits of output even on \bitwidth{32}
platforms. This means that \code{random_shuffle} will malfunction on sequences
larger than 32K objects. This is a problem that needs to be resolved. The
solution, probably, will be to provide \bitwidth{32} random number generators
as an option. \note{Has it already been done?}

\subsection[\texttt{sort}]{\code{sort}}

The \code{sort} template is implemented using the \algorithm{QuickSort}
algorithm. This was shown to be significantly faster (over twice as fast) than
using \algorithm{HeapSort} based on the heap functions in this library. This
implementation of \algorithm{QuickSort} is recursive. Since each recursive
call has private state, it is unclear if a non-recursive version would be any
faster (at the time of this writing, no performance comparisons between
recursive and non-recursive versions have been made). Stack consumption of the
recursive implementation should be $O(\log(n))$ on the average, which is not
excessive. However, the stack consumption would be $O(n)$ in the worst case,
which would be undesirable for large $n$.
